 \documentclass[fullscreen=true, bookmarks=false]{beamer} % , fleqn
 \usepackage[utf8]{inputenc}
 \usepackage[english,russian]{babel}
 \usepackage{xcolor}
 \usepackage{amsmath}
 \usetheme{PaloAlto}

%\usepackage{bm}

\setcounter{tocdepth}{1} % подробность оглавления
\title{Краткое введение в машинное обучение}

\begin{document}

 \begin{frame}
 \transdissolve[duration=0.2]
 \titlepage
 \end{frame}

\section*{План}

\begin{frame}
\begin{quote}
При изучении наук примеры полезнее правил\\
{\em Исаак Ньютон}
\end{quote}

\begin{quote}
Нет царских путей к геометрии\\
{\em Евклид}
\end{quote}
\end{frame}


% выводим оглавление
\begin{frame}
 \transdissolve[duration=0.2]
 %\frametitle{Содержание}
 \tableofcontents[]
\end{frame}


\section{Задача оптимизации}

%%%%%%%%%%%%%%%%%%%%%%%%%%%%%%%%%%%%%%%%%%%%%%%%%%%%%%%%%%%%%%%%%%%%%%%%%%%%%%%

\begin{frame}{}
 \frametitle{Задача про домики}
Небольшая компания производит домики для кошек. Фиксированные издержки составляют 20 тыс. руб. в месяц. Переменные издержки составляют 3 тыс. руб. на каждый проданный домик. В первый месяц по цене 6 тыс. руб. за домик было продано 100 домиков. Во второй месяц была установлена цена 8 тыс. руб. за домик, и не было продано ни одного домика. В предположении, что спрос линейно зависит от цены, определите оптимальную цену и объем продаж.
\end{frame}

%%%%%%%%%%%%%%%%%%%%%%%%%%%%%%%%%%%%%%%%%%%%%%%%%%%%%%%%%%%%%%%%%%%%%%%%%%%%%%%

\begin{frame}{}
 \frametitle{Задача про домики}
 
%\begin{figure}[]
%\includegraphics[]{HolesExperiment1} 
%\label{fig:HolesExperiment1}
%\end{figure}
Пусть $x$ -- количество проданных домиков, $Q$ -- прибыль компании
\begin{gather*}
\nonumber
 c(x) = 20 + 3x\\
\nonumber
 x = 50(8-p)\\
\nonumber
 p(x) = 8-0.02x\\
\nonumber
 Q(x) = p(x)x - c(x)\\
\nonumber
 Q(x) \rightarrow max
\end{gather*}

\end{frame}

%%%%%%%%%%%%%%%%%%%%%%%%%%%%%%%%%%%%%%%%%%%%%%%%%%%%%%%%%%%%%%%%%%%%%%%%%%%%%%%

\begin{frame}{}
 \frametitle{Градиентный спуск}
$L(x)$ -- функция потерь (Loss function) 
\begin{gather*}
\nonumber
L(x) \rightarrow min\\
\nonumber\\
\nonumber
\text{update step:} x = x - \lambda grad(x)\\
\nonumber\\
\nonumber
grad(x) = \frac{L(x+\varepsilon) - L(x - \varepsilon)}{2\varepsilon}
\end{gather*}

\end{frame}

%%%%%%%%%%%%%%%%%%%%%%%%%%%%%%%%%%%%%%%%%%%%%%%%%%%%%%%%%%%%%%%%%%%%%%%%%%%%%%%

\begin{frame}{}
 \frametitle{Минимизация функции двух переменных}
$L(x_1, x_2)$ -- функция потерь 
\begin{gather*}
\nonumber
L(x_1, x_2) \rightarrow min\\
\nonumber\\
\nonumber
x_1 = x_1 - \lambda grad_1(x_1, x_2)\\
\nonumber
x_2 = x_2 - \lambda grad_2(x_1, x_2)\\
\nonumber\\
\nonumber
grad_1(x_1, x_2) = \frac{L(x_1+\varepsilon, x_2) - L(x_1 - \varepsilon, x_2)}{2\varepsilon}\\
\nonumber
grad_2(x_1, x_2) = \frac{L(x_1, x_2+\varepsilon) - L(x_1 , x_2 - \varepsilon)}{2\varepsilon}
\end{gather*}

\end{frame}

%%%%%%%%%%%%%%%%%%%%%%%%%%%%%%%%%%%%%%%%%%%%%%%%%%%%%%%%%%%%%%%%%%%%%%%%%%%%%%%

\section{Задача классификации}

%%%%%%%%%%%%%%%%%%%%%%%%%%%%%%%%%%%%%%%%%%%%%%%%%%%%%%%%%%%%%%%%%%%%%%%%%%%%%%%

\begin{frame}{}
 \frametitle{Задача классификации}

\end{frame}


\end{document}